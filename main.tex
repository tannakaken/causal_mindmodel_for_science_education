\documentclass[12pt, unicode]{beamer}
\usepackage{luatexja}
\usepackage{xurl}
\usepackage{tikz}
\usetikzlibrary{
  intersections,
  calc,
  arrows.meta,
  angles,
  quotes,
  graphs,
  positioning
}
\usetheme{metropolis}
\usepackage[absolute,overlay]{textpos}
\title{理科教育における実験の意義 \\ メンタルモデルと因果}
\author{淡中 圏}
\date{}

\begin{document}

\frame{\titlepage}

\begin{frame}{要約}
\begin{itemize}
\item 理科教育における実験の意義について考える
\item \textbf{「因果的なメンタルモデル」}を構築することが重要なのではないか、と提案する
\item そのためには\textbf{「生徒を因果的に現象に関わらせる」}ような実験が必要だと主張する
\item メンタルモデルの構築後であれば、座学やシミュレーションをより有意義にすることができるとも主張する
\end{itemize}
\end{frame}

\begin{frame}{目次}
\begin{itemize}
\item 理科教育における実験の意義 4P
\item 補遺A:メンタルモデル 24P
\item 補遺B:因果関係とは何か 32P
\item 補遺C:脳内のモデル 52P
\item 参考文献 65P
\end{itemize}
\end{frame}

\begin{frame}
{\LARGE 理科教育における実験の意義}
\end{frame}

\begin{frame}{「実感」とは何か?}

科学教育などで「実験の役割」について考える際、

\begin{itemize}
\item まずは興味を持ってもらう
\item 実感を持ってもらう
\end{itemize}

などの説明がされることが多い。

しかしこの場合の「実感」とはなんであろうか?

それは重要なものであろうか?

実感を持つのに実験は必須なのであろうか? 実験以外に実感を持たせることはできないのであろうか?

それについて少し考えてみたい。

\end{frame}

\begin{frame}{仮説:「実感」の正体はメンタルモデル}

「実感」という言葉をもう少しわかりやすくするために、別の概念で言い換えてみよう。

それはメンタルモデルである。

これは、動物が何かを予測し、そして欲しい結果を得るために行動するときに「世界のメカニズム」の簡略化された模型(モデル)を
心の中に持つということを意味している言葉だ。

これが心の中にあることで「こうなったらこうなるはず」と予測し、そして「こうするためにはこうすればいい」と欲しい結果を得るための行動を選ぶことができる。

\end{frame}

\begin{frame}{脳は各瞬間に予測する器官である}

ある程度神経系を発達させた動物は餌を得て繁殖するために必要なものを世界の中で求め、危険から逃れるために、全ての瞬間に先を予測している。

僕らが冷蔵庫の中を開けて食べたかったプリンがなくなっていたら驚くのは「冷蔵庫の中にプリンがあるはず」と予測しているからで、
それは「冷蔵庫の中のプリン」というメンタルモデルを持っているから。

\end{frame}

\begin{frame}{びっくりするのは脳が予測しているからである}

ぼうっと階段を上がっている時に、存在しない段を登ろうとしてびっくりするのも階段についての間違ったメンタルモデルを持ってしまったから。

私たちの脳は全ての瞬間にメンタルモデルを作っては消し、それによって先を予測する。私たちが「びっくりすること」ができるのは
私たちが予測をするからで、予測ができるのはメンタルモデルを作っているからである。

\end{frame}

\begin{frame}{ある種のメンタルモデルは再利用できる}

学問の役割は再利用できるメンタルモデルを脳内に必要な精度で構築するための「外部足場」を提供することにある。

物理学は世界の動き方についての基本的なメンタルモデルを提供する。

高校の物理学のもたらすメンタルモデルは、それ以前の(アリストテレスが文章化したような)日常的なメンタルモデルよりも広い適用範囲を持ち、
とても有用だ(量子力学や相対性理論など、より正確な理論が提供された現在でも、古典物理が適用できる範囲は広い)。

重要なことは「理論は人間の精神内にメンタルモデル化されないと使えない」ということである。

\end{frame}

\begin{frame}{高校物理をメンタルモデル化するためには?}

理論を構成する主張や数式はメンタルモデルを持っていれば有用だが、
メンタルモデルをまた持たない人にはうまく使えない。

人の心の中にメンタルモデルを構築する戦略を練らなくてはいけない。

\end{frame}

\begin{frame}{新しいメンタルモデルの導入は難しい}

人々は、世界を生きるためにすでになんらかのメンタルモデルを持っている。
高校物理のもたらすメンタルモデルはその日常的なメンタルモデルと矛盾している。

例えば「雪だるまにセーターを着せると雪だるまはどうなるか?」という質問に我々はつい
「セーターを着ると暖かくなる」と日常的なメンタルモデルで考えがちである。

より科学的なメンタルモデルを使うのには訓練がいる。

\end{frame}

\begin{frame}{複数のメンタルモデルの同居}

また我々は複数の矛盾したメンタルモデルを使い分けている。
古いメンタルモデルは決して消えない。
そして難しい問題を解いている最中に、すでに捨てたはずの古いメンタルモデルが現れることが
あることが知られている。

日常的なメンタルモデルの限界を意識させ、より科学的なメンタルモデルを導入し、
日常的なメンタルモデルと同居させることにもっと戦略的になる必要があるのではないか?

\end{frame}

\begin{frame}{因果的なメンタルモデル}

私たちが日常で一番使うメンタルモデルは「因果的なメンタルモデル」つまり「AがBに影響を与え、BがB'になる」というような
世界の中の事物が影響を与え合い、原因が結果をもたらすメカニズムを描写している(数学が難しい理由は、それが因果的ではないからかもしれない)。

\end{frame}

\begin{frame}{因果的なメンタルモデルが重要な理由}

動物がそもそもメンタルモデルを持つ理由は「自分が世界に影響を与えるため」「世界から自分への影響を予測するため」であった。

なので「因果的メンタルモデル」が我々にとって重要であるのは進化論的にも自然な話である。

そして我々がそのような「因果的メンタルモデル」を学ぶのは「実際に自分が与えた世界への影響を観測する時」である。
つまり「状況に自分が埋め込まれている場合」に最も「因果的メンタルモデル」を我々は学習できる。

\end{frame}

\begin{frame}{実験によってまずは興味を持ってもらう}

実験をする理由として「まずは興味を持ってもらう」ことは重要である。
人は「自分が埋め込まれている状況についての物事」に興味を持つ。
まだ実感を持っていない(メンタルモデルを持てていない)理論に関する主張や数式に興味を持つことは難しい。

そこで、実際に目の前にすること、そして予想と違う現象が起こること
(日常的なメンタルモデルでは説明できない現象を観察すること)により、興味を持ってもらうことは可能である。

ただこれ自体は、現在の技術では動画(将来的にはVRやAR)などである程度代用可能なものではある。

\end{frame}

\begin{frame}{実験が重要な理由}

実験が本当に重要なのは、生徒が「自分が状況に組み込まれている状況で学習すること」である。
つまり現象に対して生徒が因果的に関わる、より平たく言えば、生徒の行動が現象に対して影響を与えることで、
生徒は現象に対してメンタルモデルを持つ可能性ができる。

運よく正しいメンタルモデルが構築されれば、生徒はそれと理論の主張や数式を組み合わせることができ、
理論の主張や数式を使って、その後の現象の推移を予測したり、目的の現象を起こすように介入したりすることができる。

\end{frame}
\begin{frame}{メンタルモデルを定着させる}

長期記憶に保存されて、何度も使われたメンタルモデルは、そうは忘れない。大雑把な把握は細部が忘れた後も残る。
そこまでいけば、生徒は理論の細部を忘れても、教科書やWebなどで細部を自分で調べ直すことにより、
理論を復元することができる。

ここまで持っていく端緒を作ることが実験で一番重要なことではないだろうか?

\end{frame}

\begin{frame}{実験だけではダメな理由}

しかし多くの実験をそのような目的で使うためには、通常の実験よりも多くの準備を必要とする。

生徒が実験のパラメータを操作することが可能でなくてはいけない。

決まったパラメータで行われるだけの実験は、最初の興味は引けても、何の因果的なメンタルモデルの構築も従わない可能性が高い。

重要なことは\textbf{「相関関係は因果関係ではない」}ことである。

「Aが起こった後にBが起こった」は単なる相関関係であって、因果関係ではない。

因果関係のためには「\textbf{私}がAを起こす」という介入が必要である(詳しくは補遺)。

\end{frame}

\begin{frame}{必要な実験の手順}

因果的なメンタルモデルを構築するためには

\begin{enumerate}
\item 初期パラメータで実験をする
\item 予測させる
\item 結果を見る(興味を惹かせる)
\item パラメータを変更させる(パラメータの操作の範囲は教師が与える)
\item 予測させる
\item 結果を見る
\item 目的を与える
\item パラメータを調節させる(理論の主張や数式など結果を予測し目的のためのパラメータを調節させるための道具は教師が与える)
\end{enumerate}

などの手順が必要である。
\end{frame}

\begin{frame}{シミュレーションも大事}

我々は日常の中で実際に大量の因果的メンタルモデルを作っては破棄し、一部は長期保存する。
その多くは実際に経験することで得ているが、実は全てではない。

私たちは日常的に「シミュレーション」している。
そもそもメンタルモデルは我々が予測、すなわちシミュレーションするためのものだ

メンタルモデルを使ってシミュレーションすることによって、我々は実験をせずに
新しいメンタルモデルを構築したり、間違ったメンタルモデルを破棄したりして、
世界から新しい知識を得ることができる。

\end{frame}
\begin{frame}{シミュレーションの例}

例えば次のような思考実験をしてみる。

「燕が低く飛んだらの雨」という主張について考えるときに、
雨の日に燕が低く飛ぶ理由を知らなくても我々は「燕が雨の原因」だとは考えない。
それは実はすでにある程度の燕や天気に関する因果的メンタルモデルを我々は持っているから、
そんなことはあり得ない、とわかるのである。
少し不自然な例を出せば、例えば「雨の日に無理やり燕を高く飛ばしたら晴れるだろうか?」という問いを考える(脳内でシミュレーションする)ことで、「燕が雨の原因」という仮説を
実際に実験することなく否定できる。

実験を全て脳内シミュレーションに置き換えることはできないが、ある程度メンタルモデルが構築されていれば、脳内シミュレーションでかなりの部分が代替できる。

\end{frame}

\begin{frame}{なぜ実験をしなくても実感が湧く人がいるのか}

理論の全ての細部について実験をしてメンタルモデルを構築する必要は実はない。

ある程度の基本的なメンタルモデルを構築していて、現象を予測したり制御したりできるようになっていれば、
あとは脳内及び紙の上やコンピュータ内でシミュレーションをすることで、メンタルモデルを精緻化することができる。

それが座学の役割であり、実際に全てを実験することのコストを考えればこれはものすごく重要である。

また、それらの中間として、コンピュータによるインタラクティブなシミュレーションも、全ての実験を代替することはできないものの、それに近い体験を提供するものとして重要である。

\end{frame}

\begin{frame}{因果関係への注目}

実は因果関係は哲学的にさまざまな議論がある概念なので、
科学の中では語りにくい概念であった(詳しくは補遺B)。

なので教育学的にも因果関係に強く注目した議論はあまりないと思われる。

科学哲学的に及び統計学的に因果関係とは何か、という部分は補遺Bに回すとして、
ここまでが「メンタルモデルと因果関係に注目することによる科学教育への提言」である。

\end{frame}

\begin{frame}
{\LARGE 補遺A:メンタルモデル}
\end{frame}

\begin{frame}{補遺A:メンタルモデル}

メンタルモデルという概念は1943年に哲学者ケネス・クレイクが考案し、
40年間あまり研究されず、1983年にフィリップ・ジョンソン=レアードがワーキングメモリ(作業記憶)内に
構築されたものとしてのメンタルモデルを、同年にデドル・ジェントナーとアルバート・スティーブンスによって
長期記憶内に構築されたものとしてのメンタルモデルを考察した。

これら二つは矛盾するものではなく、相補的なものとして捉えられる。

その後、UIの研究などで注目を集めていく。

\end{frame}

\begin{frame}{補遺A:メンタルモデル}

UIでは、ユーザーのメンタルモデルが実際の挙動とあっていない時に、UIは使いにくくなると考えられる。

そのために、使いやすいUIはユーザーのメンタルモデルに合わせるか、
もしくはユーザーに挙動にあったメンタルモデルを受け入れさせる必要がある。

科学の場合はどうしても後者にならなくてはいけない。

\end{frame}

\begin{frame}{補遺A:メンタルモデル}
人々は何かのメンタルモデルを身につける前に、元から別のメンタルモデルを持っていることが多い。

市川伸一は、渦に巻いたホースから出る水の動きを子供に予測させると渦を巻いて出ていくと予測することが多いという
実験結果を紹介している。

子供や動物もなんらかの保存量(この場合は「渦を巻く」という性質が保存している)を探すようであるが、
それが物理学的な保存量と一致する(この場合は運動量が保存するので直進すると予測する)ことはほぼない。

\end{frame}

\begin{frame}{補遺A:メンタルモデル}
元科学教師の哲学者ガストン・バシュラールは浮力の実験を見た生徒が
「周囲の水が物体を押す」という力学的な視点より
「物体が浮き上がろうとしている」という日常的な視点を持ちがちなことに触れて
「空っぽな脳に正しい理論を入れる、という考え方では上手くいかない。すでに間違った理論が入っているのだから」
と述べている。

理論より世界観を重んじるバシュラールの科学観はその後のクーンのパラダイム理論にも近い。

\end{frame}

\begin{frame}{補遺A:メンタルモデル}

イガル・ガリリとヴァルダ・バーの二人は、力学を学んだ学生に対する実験で、
身近な問題では力学を上手く使えても、複雑な問題で日常的なメンタルモデルに回帰してしまいがちなことを示した。

我々は複数の矛盾したメンタルモデルを同居させている。
その使い分けをするためには、「自分の今使っているメンタルモデルは何か? メンタルモデルの限界を超えていないか?」
という反省が必要だが、そのような自己反省テクニックを教える教育は今のとこなされていない。

\end{frame}

\begin{frame}{補遺A:メンタルモデル}

メンタルモデルは特定の問題解決のための道具であるが、実際にはそれ以上に使えるものである。

何か新しい問題を見た時、我々はすでに知っているメンタルモデルと似ているところを見つけることで、
新しいメンタルモデルの構築をスムーズにすることができるし、
今まで持っていたメンタルモデルの洗練化もできる。

これが「比喩」「メタファー」「アナロジー」と呼ばれるものである。

多くのメンタルモデルと適切な比喩のスキルを持つことで、新しい問題への対処能力が上がる。

\end{frame}

\begin{frame}{補遺A:メンタルモデル}

なんらかのメンタルモデルを持つことにより、その他のスキルも上がることが期待されることがある。
例えば数学やプログラミングを覚えることで、論理能力が上がることが期待されている。
しかしチェスの能力が他のゲームの能力に影響したり、プログラミングの能力が一般の論理能力に影響することは、
調査による限りなさそうなのが実際のところだ。

しかし個人的な感触を言えば、たくさんのメンタルモデルをもち、
適切な比喩の技術(類似を見つける技術と、類似を批判する技術)を持っていれば、
ある分野で学んだことを他の分野に応用することは可能であるように感じられる。

そのためにも高校科学のための適切なメンタルモデルをどう構築するかは、大切な話題だと思われる。

\end{frame}

\begin{frame}
{\LARGE 補遺B:因果関係とは何か}
\end{frame}

\begin{frame}{補遺B:因果関係とは何か}

因果関係(原因と結果)の関係は日常の中で普通に出てくる。

例:
\begin{itemize}
\item 塩を入れすぎたから、スープが美味しくなくなった。
\item 言いすぎたから怒らせちゃった
\item 野菜が高いのは、雨が振りすぎたせいだ
\item などなど
\end{itemize}

\end{frame}

\begin{frame}{補遺B:因果関係とは何か}
でも「因果関係とは何か?」と真面目に考えると実はものすごく難しい!
\end{frame}

\begin{frame}{補遺B:因果関係とは何か}
ヒュームの観察

ボールがぶつかって他のボールが動くとき、我々はそこに「ボールがぶつかったから他のボールが動いた(ボールが動かされた)と考えるが、
実際に観察されたのは、「ボールがぶつかる」「他のボールが動く」という二つの現象が相次いで起こっただけ(相関関係)で、因果関係自体は観察されていない。
実際、「ボールがぶつかる」のと「他の原因で他のボールが動く」が相次いで起こったことを否定する観察はここにはない。
観察から因果関係は導けない。ではなぜ我々はそこに因果関係を見るのか。
それは我々の思考の癖でしかない。
\end{frame}

\begin{frame}{補遺B:因果関係とは何か}
因果関係を哲学的に分析する方法は複数提案されている(詳しくは『理系人に役立つ科学哲学』に複数提示されている)。
その中で、「可能世界」という道具を使ったものを紹介する。

「AがBの原因である」とは「Aが起きていれば\textbf{必然的に}Bが起きるが、Aが起きていなければBが起きるのは不可能である」ということ。
「必然的」とは「この世界とよく似た可能な世界の全てで」という意味。
「可能」とは「この世界とよく似た可能な世界のどこかで」という意味。
このような考えを\textbf{可能世界}という。

\end{frame}
\begin{frame}{補遺B:因果関係とは何か}

可能世界を使った分析のメリットとデメリットを挙げる。

メリット: 形式化しやすい。数学的な概念にもしやすい。

デメリット: 「可能世界」などの概念は、「形而上」つまり「この世界の外側にある」ので科学で扱いにくい(というか扱えない)。

\end{frame}

\begin{frame}{補遺B:因果関係とは何か}
因果「塩を入れすぎたから、スープが美味しくなくなった」を可能世界で説明すると
「塩を入れていないこの世界とよく似た可能世界では、スープは美味しい」となる。

「事実と異なる世界」を推論しているので、このような命題に関する推論を「反事実推論」と呼ばれる

哲学者のデイビッド・ルイスはこのような可能世界が実際に存在すると主張し、
これによって因果関係などを説明しようとした。

ルイスの哲学は科学の中に収めるのが非常に難しい。

\end{frame}

\begin{frame}{補遺B:因果関係とは何か}
因果関係に関するここまでの簡単なまとめ
\begin{itemize}
\item 因果関係自体は観察不可能。いくら現象を観察しても因果関係は出てこない
\item 「因果関係とは何か」を概念的に考察するのも難しい。
\end{itemize}

このような理由で、科学では因果関係について考えることはほとんどなくなっていった。

\end{frame}

\begin{frame}{補遺B:因果関係とは何か}

統計学では遺伝学者のゴルトンが、「平均への回帰」をまず因果関係として説明しようとした歴史がある。
その後ゴルトンは「平均への回帰」が単なる確率論的な現象で、因果関係なしでも説明できることに気づいた。

ゴルトンの弟子のピアソンは統計学から因果関係を取り除いて相関関係だけに注目する潮流を作った。

それは当時の統計学をはじめとした科学全体が「客観性」を重んじていたからである。

因果関係と違って相関関係はデータだけを見ればわかったからである。

\end{frame}

\begin{frame}{補遺B:因果関係とは何か}

しかし、ピアソンは統計学から因果関係を取り除き切ることができなかった。
疑似相関というピアソンの用語がそれを物語っている。

疑似相関も相関関係なのになぜ疑似がつくかといえば、
「疑似因果」と言えばわかりやすいところを、「因果」という言葉が使いたくなかったから、
疑似相関なのである。

ピアソンも因果関係でない相関関係と因果関係には違いがあると考えたのだが、
因果という言葉を使わずにその違いを説明することはできなかった。

\end{frame}

\begin{frame}{補遺B:因果関係とは何か}

フィッシャーは実験計画によって因果関係を実証する方法を見つけた。
それでも統計学や科学では因果について語ることはあまり好まれなかった。

池田清彦は「科学の目的は相関関係を調べることで因果関係を調べることではない」と断言している。

しかしそれでは、「科学捜査」の意味がわからないし、「歴史学」「地球惑星科学」「宇宙論」「進化生物学」など
多くの科学が科学ではないことになってしまう。
これらの分野は相関関係ではなく因果関係について調べているからである。

\end{frame}

\begin{frame}{補遺B:因果関係とは何か}

ジューディア・パールは、反事実推論を統計学の中で扱う手法を開発した。
それが「統計的因果推論」と呼ばれる技術で、この技術の導入によって統計学では「因果革命」が起きていると言われる。

統計的因果推論によれば「塩を入れていないこの世界とよく似た可能世界では、スープは美味しい」を証明するために行うべきことは、
「塩を入れない以外全く同じスープを作ること」である。
もちろん全く同じは無理であるが、スープを作るのが自分で、
事実上問題にならないくらい違いを小さくコントロールできると自分で信じられるならば、問題にならない。

\end{frame}

\begin{frame}{補遺B:因果関係とは何か}

つまりパールは因果関係を示すためには、世界をただ観察するだけではなくて、世界に対して介入しなくてはいけないと主張した。
実際、フィッシャーの実験計画が上手くいく理由もパールの理論で説明できる。

またパールの理論では、推論する前に世界の因果的なモデルが必要とされる。
そのモデルがあるから「事実上問題にならないくらい違いを小さくコントロールできる」ことが信じられるのだ。

これが従来の統計学で受けいられなかったのは、パールの理論に含まれる世界への介入、および事前に持っているモデルが、
統計学が信奉してきた「客観性」に反するからである。

\end{frame}

\begin{frame}{補遺B:因果関係とは何か}

またパールの理論を使えば、必ずしも実験は必須ではない。

ある程度信頼のおけるモデルを持っていれば、そのモデルを使って「反事実的な推論」すなわち「シミュレーション」をすることにより、
さまざまな現象を予測したり、現象の原因か何かを推論することができる。
(この議論が本題におけるシミュレーションの議論、実験が必須かどうかの議論とほぼ同じものであることに注目)

統計的因果推論の技術により、二酸化炭素などが実際どれくらい気候変動の原因になっているかを計算する研究が進行中である。
これらの分析をするためには因果の概念が欠かせない。

\end{frame}

\begin{frame}{補遺B:因果関係とは何か}
パールは因果に関する推論ができるのは人類だけだと考えている。
しかしメンタルモデルの議論により、むしろ多くの動物もメンタルモデルを使ってシミュレーションを行い、
因果に関する推論ができていると考える方が自然である。

しかし人類も含めた動物が正しく推論できるのは、手に触れて、
簡単に介入できるものに限られている(因果の理解のためには介入が必要、ということを考えれば、自然である)。

人間がデータの中に間違った因果を見がちなのも、そのような手に触れないものの因果の推論は、
人間にとっても大変なことの証拠である。

\end{frame}

\begin{frame}{補遺B:因果関係とは何か}

むしろ人間の持つ素朴な因果的推論と統計学の高度な数理的推論を組み合わせることができるようになったことが
「因果革命」の本質と考えた方が筋が良いように思われる。

今までメンタルモデルで行っていた因果推論を数理モデルで行うわけである。

そしてこれが実際に人間に使えるものになるためには、完全に正確なものではなくてもいいので、
適切な正確さでそのような数理モデルのメンタルモデルを脳内に構築して、
それを操作して、世界に介入することができなくてはいけないのである。

\end{frame}
  
\begin{frame}{補遺B:因果関係とは何か}

因果関係自体ではないが、それに関連する重要な哲学としてイアン・ハッキングの『介入主義』がある。
これは、「因果的に介入できるものは実在していると考えてかまわないのではないか?」という実在論に関する考え方である。

\end{frame}

\begin{frame}{補遺B:因果関係とは何か}

科学の歴史では熱素や燃素、エーテルなど、かつてあると考えられたけど結局存在しなかったものは多い。
なので科学哲学的には「現在存在することになっているからといって存在すると言っていいのか?」という疑問が現れる。

ハッキングは「電子を油滴にぶつけてクオークの電荷を調べる」という実験を見て、
「電子に直接因果的に働きかけて欲しい結果を得ている以上、電子はかなり実在していると見て良い」と考えた。
歴史的にもこのように直接の因果的な働きかけが科学的実験として成功したあとに、
存在が否定されたものはない(今後現れないとは限らないが)。

よってハッキングの考えによれば、電子の実在はかなり信頼できるが、クオークの実在はそれほど信頼できないことになる。

\end{frame}

\begin{frame}{補遺B:因果関係とは何か}

ハッキングは理論よりもこのような「因果的に介入できる物」に関しての方が信頼できると考えている。
なぜなら理論は世界全体に関して語るものであるから、多くの物事について語っており間違っている危険性が高い。

それに対して、因果的な介入されているものは「今」「ここで」起こっていることなので、関係物が少ないのだ。
だから間違いにくい。

\end{frame}

\begin{frame}{補遺B:因果関係とは何か}

これは本文において、なぜ実験で生徒が現象に因果的に介入することが重要なのか、という議論と関係が深いように思える。

つまり、生徒にとって理論は「世界全体」という想像しにくい身近なものだが、自分が因果的に介入できる現象は
目の前の身近なものなので、より実感が湧きやすい、つまり「メンタルモデルが作りやすい」ものになるのだ。

そのようなものに関して間違いにくいことを、科学の歴史も証明している、と考えることもできる。

\end{frame}

\begin{frame}
{\LARGE 補遺C:脳内のモデル}
\end{frame}

\begin{frame}{補遺C:脳内のモデル}

本文と補遺Aではまるでメンタルモデルが脳内にあるように書いた。
またパールも因果的推論をするために脳内にモデルがあると想定している。

実際問題脳がどのようにメンタルモデルを作っているのか、わかっているわけではない。

それについて調べた範囲のことについて書いておく。

\end{frame}

\begin{frame}{補遺C:脳内のモデル}

AI開発者で脳科学者のジェフ・ホーキンスは「千の脳理論」という理論を提唱していて、
彼はどうやら格子細胞(物体の位置関係を記憶する細胞)と場所細胞(その場所に何があるかを記憶する細胞)が鍵だと考えている。

この二つによって、いくつもの地図が互いにリンクしている構造を脳内に作ることができる(ホーキンスは物理用語を転用してそれを「reference system」と呼んでいる)。
ホーキンスによれば、動物はreference systemによって「次は何が見えるか?」「指先に次は何が触るか?」を予測し続けている。
予測しているから間違っていればびっくりする(本文の議論を参照)。

\end{frame}

\begin{frame}{補遺C:脳内のモデル}

ホーキンスはreference systemは「繋がり」と「動き」を持っているので、
空間的なものだけでなく、任意の抽象的な概念のモデルを構築できると考えている。

具体的なことはまだわかっていないがホーキンスは数学的概念や曖昧な哲学的・人文的概念などもreference systemを使って実現されているはずだと考えている。

\end{frame}
\begin{frame}{補遺C:脳内のモデル}

ホーキンスが正しければメンタルモデルはまさにreference systemを使って構築されているはずである。

そう考えると、私たちが「考えが近い」「遠い」など、空間的でないものに空間的比喩を使うことが多いことや、
図解すると理解しやすくなったり、理解したふりがしやすくなったりする理由もわかる。

\end{frame}

\begin{frame}{補遺C:脳内のモデル}

ホーキンスの本からとってきたものではないが、ホーキンスの説を補強するものとして海馬に関する実験が挙げられる。

人間において海馬は「短期記憶を長期記憶に変換する」機能に関係が深いとされている。

そしてネズミの実験で、海馬の機能を制限されたネズミは「視点によらない空間のモデル」の構築が阻害されることが知られている。

\end{frame}
\begin{frame}{補遺C:脳内のモデル}

濁った水の下に足場のあるプールにネズミを落とす場合、何回か落とせばネズミは足場の位置を覚える。
そして落とす場所を変えても足場の場所に泳ぐ。

これはネズミが「視点によらない空間のモデル」を獲得していることを意味する。

\end{frame}
\begin{frame}{補遺C:脳内のモデル}

海馬の機能を制限されたネズミは落とす場所が同じなら足場の方へ泳げる(右に泳げばいい、ということは記憶できる)。

しかし、落とす場所を変えたら、足場の方へは泳げない。

これは海馬が空間のモデルを作ることに関係していると考えられる。

そのような海馬が人間の長期記憶に対して重要な機能を果たしているのは、
全てのメンタルモデルが空間のモデルを作るための機能で作られるというホーキンスの説の傍証になりうると考えられる。

\end{frame}

\begin{frame}{補遺C:脳内のモデル}

パールもホーキンスも脳内にかなり明確なモデルがあるような言葉使いをする。

しかし、そもそもそんな明確なモデルは必要ないかもしれない、と考えるのが哲学者のアンディ・クラークである。

クラークは実は脳内にあるのは、かなり粗いモデルで、モデルのほとんどは脳の外にあると考えている。

例えば、マグカップの正確なモデルが脳内になくても私は困らない。左を見ればマグカップが実際に見えるのだから。

このように、脳の外にあるもので代用できるなら、脳内にモデルを作る必要は必ずしもない。

\end{frame}

\begin{frame}{補遺C:脳内のモデル}

クラークはそこから、人類の強力な知性は、人類の脳自体の能力というより、
脳の外に知性を助ける人工物をたくさん作って、それを利用する能力だと考えた。
クラークはそれを「外部足場」という。

例えば、多くの人は2桁の掛け算が暗算でできないが、紙と鉛筆があれば3桁以上の掛け算だってできる。
これは、紙・鉛筆・数字・筆算のシステム、という人工物を外部足場とすることが我々の脳にできるからである。

そして外部足場を洗練させることにより、脳自体はほぼ変わっていないのに、
我々は先祖よりずっと高い知性を示すことができる(この100年でもIQは上がり続けている)。

\end{frame}

\begin{frame}{補遺C:脳内のモデル}

そう考えると、例えば、高校物理のためのメンタルモデルを生徒の頭に構築するとは、決して高校物理の理論を全て脳内に流し込むことにはならないはずである。

メンタルモデル自体は、ずっと小さくて複数でしかもお互いに矛盾しているもので構わない。
おそらく、それはいくつかの主張や図や実験の記憶の断片の集まりになるかもしれない。

それを、正確なことが書いてある教科書やWebサイト、思い出したいときに復習に使える典型的な練習問題と組み合わせることにより、
必要なときに必要なだけ精緻化して使えるようになるのではなかろうか。

\end{frame}

\begin{frame}{補遺C:脳内のモデル}

認知言語学者のレイ・ジャッケンドフは、言葉(言葉も文字もアンディ・クラークのいう人工物の一種である)が
思考を助けるのは、言葉が思考の「取っ手」となるからだという。

全体を覚えていなくても、連想の起点となる言葉を覚えておけば、他の言葉やイメージが芋づる式に思い出される。

この考えとアンディ・クラークの考えを組み合わせれば、起点となるものは非常に小さくても、
脳内の他の部分や、
脳の外にある知識へとそこから繋げていく手法を身につければ、脳に負担をかけずにかなりのことができるようになるはずなのである。

\end{frame}

\begin{frame}{補遺C:脳内のモデル}

最近では「脳は予測する器官」という考え方が発展して、脳の全てをそこから理解しようとする理論も生まれている。
それは「自由エネルギー原理」と呼ばれていて、「予測して修正」を繰り返す脳の動きを
「エネルギーが小さい軌道へ修正していく」という「変分法」のアナロジーで捉える考え方である。

ここでも「物理」のメンタルモデルがアナロジーと介して他の分野に影響を与えていることが観測される。

\end{frame}

\begin{frame}{参考文献(科学関連)}
\begin{itemize}
\item 因果推論の科学(ジューディア・パール, ダナ・マッケンジー)
\item 考えることの科学―推論の認知心理学への招待(市川伸一)
\item プログラマー脳~優れたプログラマーになるための認知科学に基づくアプローチ(フェリエンヌ・ヘルマンス)
\item エンジニアの知的生産術 ―効率的に学び、整理し、アウトプットする(西尾 泰和)
\item 脳は世界をどう見ているのか: 知能の謎を解く「1000の脳」理論(ジェフ・ホーキンス)
\item 脳の大統一理論: 自由エネルギー原理とはなにか(乾敏郎, 阪口豊)
\end{itemize}
\end{frame}

\begin{frame}{参考文献(哲学関連)}
\begin{itemize}
\item 理系人に役立つ科学哲学(森田邦久)
\item 現れる存在: 脳と身体と世界の再統合(アンディ・クラーク)
\item 表現と介入: 科学哲学入門(イアン・ハッキング)
\item 構造主義科学論の冒険(池田晴彦)
\item デイヴィッド・ルイスの哲学(野上志学)
\end{itemize}
\end{frame}

\end{document}
